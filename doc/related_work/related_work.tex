\documentclass[10pt,a4paper]{article}
\usepackage[margin=1in]{geometry}
\usepackage{listings}
\usepackage{amsmath}
\usepackage{amsthm}
\usepackage{amsfonts}
\usepackage{enumerate}
\usepackage{amssymb}
\usepackage{graphicx}
\usepackage{booktabs}
\usepackage{xcolor}
\usepackage{stackengine}
\usepackage{tabularx}
\usepackage{xcolor}
\usepackage{hyperref}

\lstset{
  basicstyle= \footnotesize \ttfamily,
  % numbers=left,
  % numberstyle=\small,
  % numbersep=8pt,
  showstringspaces=false,
  breaklines=true,
  frame=lines,
  columns=fullflexible,
  backgroundcolor=\color{white},
  emphstyle=\color{blue},
  keywordstyle=\color{brown},
  stringstyle=\color{red},
  commentstyle=\color{gray},
}
\setlength\parindent{0pt}

\newcommand{\<}{\langle}
\renewcommand{\>}{\rangle}
\newcommand{\norm}[1]{\left\lVert #1 \right\rVert}
\newcommand{\abs}[1]{\left\vert #1 \right\vert}
\newcommand{\prths}[1]{\left( #1 \right)}
\newcommand{\braces}[1]{\left\{ #1 \right\}}
\newcommand{\brackets}[1]{\left[ #1 \right]}
\newcommand{\chevrons}[1]{\left\< #1 \right\>}
\newcommand{\prob}[2][]{\underset{#1}{\mathbb{P}}\left[ #2 \right]}
\newcommand{\set}[2]{\left\{ #1 \; \middle\vert \; #2 \right\}}
\newcommand{\tr}[1]{\textrm{tr} \left( #1 \right)}
\newcommand{\bra}[1]{\left\< #1 \right\vert}
\newcommand{\ket}[1]{\left\vert #1 \right\>}
\newcommand{\red}[1]{\textcolor{red}{#1}}
\newcommand{\todo}{\red{?}}

\title{Related and Future Work}
% \author{Kanguk Lee}
\date{}

\begin{document}
\maketitle

\section{Related Works}

\subsection{Verification of Quantum Programs}
Several approaches have been employed for formally verifying quantum programs.
Unlike our research, which focuses on verifying an execution model of quantum
programs, these methods are used to validate a particular property of a specific
quantum program.

\subsubsection{QWIRE: A Core Language for Quantum Circuits}
$\mathcal{Q}$WIRE is a language for defining quantum circuits and an interface
for their manipulation within any classical host language. The
authors proved that every well-typed static circuits written in
$\mathcal{Q}$WIRE preserves ``$\textrm{tr}\prths{\rho} = 1$" property of density
matrices.

\subsubsection{Formal Verification of Quantum Algorithms Using Quantum Hoare Logic}
In this study, the authors formalized the theory of quantum Hoare logic to
reason about quantum programs. As an application, they verified the correctness
of Grover's search algorithm.

\subsection{Testing}
Several pieces of research on testing quantum computing platforms such as
Qiskit, including tests on physical semantic errors, have been conducted.

\subsubsection{QDiff: Differential testing of Quantum Software Stacks}
Qdiff performs differential testing across different backends and optimization
levels of quantum computing platforms. The authors employed K-S testing and cross
entropy to measure distances between statistical distributions.

\subsubsection{MorphQ: Metamorphic Testing of the Qiskit Quantum Computing Platform}
MorphQ conducts metamorphic testing on a novel set of metamorphic
transformations. The authors use K-S testing to calculate a p-value of
distribution equivalence.

\section{Future Work}
\subsection{Optimization: Avoiding Unnecessary }
\subsubsection{Avoiding Unnecessarily Splitting World}
The division each world into two doubles the cost of subsequent operations.
If through analysis of a quantum circuit reveals that some qubits remain
unaltered throught the circuit ater measurement, it becomes unnecessary to
preserve the measurement result in a classical register, as the qubit itself
retains the information of the measurement output. By eliminating the classical
bit stroing the measurement output information, the need for dividing the world
is also removed as the classical states are not impacted by the measurement.
Hence, it suffices to simply add the two resultant density matrices, each
multiplied by its measurmeent probability.
  ( From Classical to Quantum Shannon Theory (mark M. Wilde): page 159 )

\subsubsection{Representation of Matrix as a Two-Dimensional Array}
As previously noted, utilizing a function to represent a matrix is
mathematically elegant and simplifies the proof of physical consistency.
However, this design choice may not offer optimal computational performance. If
we establish physical consistency while representing the matrix as a
two-dimensional array, the resulting OCaml code will likely demonstrate superior
performance.

\subsection{Support for OpenQASM3.0}
OpenQASM3.0 supports more expresesive and complex dynamic quantum-classical
high-level interactions. Supports for OpenQASM3.0 is
in its infancy and the implementation is expected to change significantly.
(\href{https://qiskit.org/documentation/apidoc/qasm3.html#importing-from-openqasm-3}{quoted
  from the official document}) Hence, a physically-consistent execution for OpenQASM3.0 will prove
beneficial for developing support for OpenQASM3.0.

\end{document}

