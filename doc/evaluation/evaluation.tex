\documentclass[10pt,a4paper]{article}
\usepackage[margin=1in]{geometry}
\usepackage{listings}
\usepackage{amsmath}
\usepackage{amsthm}
\usepackage{amsfonts}
\usepackage{enumerate}
\usepackage{amssymb}
\usepackage{graphicx}
\usepackage{booktabs}
\usepackage{xcolor}
\usepackage{stackengine}
\usepackage{tabularx}
\usepackage{xcolor}

\lstset{
  basicstyle= \footnotesize \ttfamily,
  % numbers=left,
  % numberstyle=\small,
  % numbersep=8pt,
  showstringspaces=false,
  breaklines=true,
  frame=lines,
  columns=fullflexible,
  backgroundcolor=\color{white},
  emphstyle=\color{blue},
  keywordstyle=\color{brown},
  stringstyle=\color{red},
  commentstyle=\color{gray},
}
\setlength\parindent{0pt}

\newcommand{\<}{\langle}
\renewcommand{\>}{\rangle}
\newcommand{\norm}[1]{\left\lVert #1 \right\rVert}
\newcommand{\abs}[1]{\left\vert #1 \right\vert}
\newcommand{\prths}[1]{\left( #1 \right)}
\newcommand{\braces}[1]{\left\{ #1 \right\}}
\newcommand{\brackets}[1]{\left[ #1 \right]}
\newcommand{\chevrons}[1]{\left\< #1 \right\>}
\newcommand{\prob}[2][]{\underset{#1}{\mathbb{P}}\left[ #2 \right]}
\newcommand{\set}[2]{\left\{ #1 \; \middle\vert \; #2 \right\}}
\newcommand{\tr}[1]{\textrm{tr} \left( #1 \right)}
\newcommand{\bra}[1]{\left\< #1 \right\vert}
\newcommand{\ket}[1]{\left\vert #1 \right\>}
\newcommand{\red}[1]{\textcolor{red}{#1}}
\newcommand{\todo}{\red{?}}

% \title{Coq Implementation}
% \author{Kanguk Lee}
% \date{\today}

\begin{document}
% \maketitle
\section{Execution Example}

Here is the example of a dynamic circuit discussed in section 2.2.1, written in OpenQASM2.0:\\
\begin{lstlisting}
OPENQASM 2.0;

gate u2(phi,lambda) q { U(pi/2,phi,lambda) q; }
gate h a { u2(0,pi) a; }

qreg q0[1];
creg c0[1];
creg c1[1];

h q0[0];
measure q0[0] -> c0[0];
if(c0==0) h q0[0];
measure q0[0] -> c1[0];
\end{lstlisting}

Note that the values in the classical registers, are either 00, 10, or 11, with probabilities of
1/4, 1/4, and 1/2, respectively.
\\

\begin{lstlisting}
$ dune exec qasm example/basic.qasm
QASMCore ========================================
RotateInstr (1.570796, 0.000000, 3.141593, 0)
MeasureInstr (0, 0)
IfInstr (0, false,
RotateInstr (1.570796, 0.000000, 3.141593, 0)
NopInstr)
MeasureInstr (0, 1)
NopInstr
RESULT ==========================================
00 : 2.5000000000000011e-01
01 : 0.0000000000000000e+00
10 : 2.5000000000000000e-01
11 : 4.9999999999999989e-01
\end{lstlisting}



\section{Comparison of Result to Traditional Quantum Simulators}

In this section, we align the results of our novel implementation with those obtained from Qiskit's
entire suite of simulators. Our aim is to verify that our results are consistent with those of
well-established, widely-used simulators.

\subsection{Chi-Square Goodness-of-Fit Test}

To compare our results with those of the extant quantum simulators, we deployed the Chi-Square
Goodness-of-Fit Test, a variant of Pearson's chi-square test. This test is employed to determine if
the observed distribution of a particular execution result, which is treated as a discrete random
variable, deviates from the expected distribution. This expectation is derived from the result of
the result of our implementation.

For the reliability of a Chi-Square goodness-of-fit test, a widely accepted rule of thumb stipulates
that the expected frequencies should be no less than 5 for all potential outputs. Consequently, we
adhered to this guideline in conducting our experiment.

The p-value of the Chi-Square goodness-of-fit test is the probability of obtaining statistical results at
least as extreme as the results actually observed, under the assumption that the null hypothesis is
correct. A smaller p-value indicates that the observed data is less likely under the null
hypothesis, and at a certain threshold (often 0.05), we might reject the null hypothesis in favor of
the alternative hypothesis.

In our comparative analysis between the results of our implementation and those of Qiskit's
simulators, we formed the null hypothesis stating, "the outcomes of the Qiskit simulator conform to
the probability distribution derived from our implementation". The Chi-Square goodness-of-fit test
was consequently conducted under this presupposition.

\subsection{Compared Simulators}

We conducted a comparison of the results from twelve quantum simulators provided by Qiskit to
determine whether they align with the probability distribution derived from our implementation.

\begin{table}[h]
\centering
\begin{tabularx}{\textwidth}{|c|c|X|}
\hline
\textbf{Quantum Simulator} & \textbf{Explanation} & \textbf{Dynamic Circuit Support}\\
\hline
\texttt{aer\_simulator} & Explanation for Simulator 1 & yes \\
\hline
\texttt{aer\_simulator\_statevector} & Explanation for Simulator 2 & yes \\
\hline
\texttt{aer\_simulator\_density\_matrix} & Explanation for Simulator 3 & yes \\
\hline
\texttt{aer\_simulator\_stabilizer} & Explanation for Simulator 4 & yes \\
\hline
\texttt{aer\_simulator\_matrix\_product\_state} & Explanation for Simulator 5 & yes \\
\hline
\texttt{aer\_simulator\_extended\_stabilizer}& Explanation for Simulator 6 & yes \\
\hline
\texttt{aer\_simulator\_unitary} & Explanation for Simulator 7 & no \\
\hline
\texttt{aer\_simulator\_superop} & Explanation for Simulator 8 & no \\
\hline
\texttt{qasm\_simulator} & Explanation for Simulator 9 & yes \\
\hline
\texttt{statevector\_simulator} & Explanation for Simulator 10 & yes \\
\hline
\texttt{unitary\_simulator} & Explanation for Simulator 11 & no \\
\hline
\texttt{pulse\_simulator} & Explanation for Simulator 12 & no \\
\hline
\end{tabularx}
\caption{Qiskit Quantum Simulators}
\label{table:qsimulators}
\end{table}

\subsection{Benchmarks}

To compare the behaviors of both static and dynamic quantum circuits between our implementation and
Qiskit Aer simulators, we executed a total of 33 benchmark programs from QASMBench[ ] and \red{all} examples
found in the OpenQASM2.0 specification.

\subsubsection{QASMBench}

We executed all the small-scale benchmarks specified in the QASMBench paper along with some of the
medium-scale benchmarks that involve 8 qubits or less, using our implementation (refer to Table 2).
This enabled us to derive the expected probability distribution for each benchmark program.

\begin{table}[h]
\centering
\begin{tabularx}{\textwidth}{|c|c|X|X|X|X|}
\hline
\textbf{Index} & \textbf{Benchmark} & \textbf{Scale} & \textbf{Qubits} & \textbf{Gates} & \textbf{CX} \\
\hline
1 & adder & small & 4 & 23 & 10 \\
\hline
2 & basis\_change & small & 3 & 53 & 10 \\
\hline
3 & basis\_trotter & small & 4 & 1626 & 582 \\
\hline
4 & bell\_state & small & 2 & 3 & 1 \\
\hline
5 & cat\_state & small & 4 & 4 & 3 \\
\hline
6 & deutsch & small & 2 & 5 & 1 \\
\hline
7 & dnn & small & 2 & 268 & 84 \\
\hline
8 & fredkin\_n3 & small & 3 & 19 & 9 \\
\hline
9 & qec\_dist3 & small & 5 & 114 & 49 \\
\hline
10 & grover & small & 2 & 16 & 2 \\
\hline
11 & hs4 & small & 4 & 28 & 4 \\
\hline
12 & inverseqft & small & 4 & 8 & 0 \\
\hline
13 & iSWAP & small & 2 & 9 & 2 \\
\hline
14 & linearsolver & small & 3 & 19 & 4 \\
\hline
15 & lpn & small & 5 & 11 & 2 \\
\hline
16 & pea & small & 5 & 98 & 42 \\
\hline
17 & qaoa & small & 3 & 15 & 6 \\
\hline
18 & qec\_sm & small & 5 & 5 & 4 \\
\hline
19 & qec\_en & small & 5 & 25 & 10 \\
\hline
20 & qft & small & 4 & 36 & 12 \\
\hline
21 & qrng & small & 4 & 4 & 0 \\
\hline
22 & quantumwalks & small & 2 & 11 & 3 \\
\hline
23 & shor & small & 5 & 64 & 30 \\
\hline
24 & toffoli & small & 3 & 18 & 6 \\
\hline
25 & teleportation & small & 3 & 8 & 2 \\
\hline
26 & jellium & small & 4 & 54 & 16 \\
\hline
27 & vqe & small & 4 & 89 & 9 \\
\hline
27 & vqe\_uccsd & small & 4 & 220 & 88 \\
\hline
28 & wstate & small & 3 & 30 & 9 \\
\hline
29 & dnn & medium & 8 & 1200 & 384 \\
\hline
30 & bb84 & medium & 8 & 27 & 0 \\
\hline
31 & qaoa & medium & 6 & 270 & 54 \\
\hline
32 & simons & medium & 6 & 44 & 14 \\
\hline
33 & hhl & medium & 7 & 689 & 196 \\
\hline
\end{tabularx}
\caption{QASMBench}
\label{table:qbenchmarks}
\end{table}

\subsubsection{OpenQASM2.0 Specification}

All the benchmark programs listed in Table 2 are static circuits. To facilitate a comparison
of dynamic circuit behaviors, we incorporated dynamic circuit benchmarks as introduced in the
OpenQASM2.0 specification (See Table 3).

\begin{table}[h]
\centering
\begin{tabularx}{\textwidth}{|c|c|X|X|X|X|}
\hline
\textbf{Index} & \textbf{Benchmark} & \textbf{S/D} & \textbf{Qubits} & \textbf{Gates} & \textbf{CX} \\
\hline
1 & inverse QFT followed by measurement & dynamic & 4 & 16 & \todo \\
\hline
2 & quantum error correction & dynamic & 5 & 6 & \todo \\
\hline
3 & quantum fourier transform & static & 4 & 36 & \todo \\
\hline
4 & quantum process tomography & static & 1 & 3 & \todo \\
\hline
5 & quantum teleportation & dynamic & 3 & 7 & \todo \\
\hline
6 & randomized benchmarking & static & 2 & 11 & \todo \\
\hline
7 & ripple carry adder & static & \red{X} & \todo & \todo \\
\hline
\end{tabularx}
\caption{OpenQASM2.0 Specification Examples}
\label{table:qbenchmarks}
\end{table}

\subsection{Experiment Result}

Presented below are the p-values derived from the chi-square goodness-of-fit test for all twelve
simulators, as compared to our implementation using all 33 benchmarks.

\textbf{\red{Image (Currently under experiment)}}
\\
\\
Even after conducting \red{10 million} shots, the p-values remain sufficiently large for us to
confidently accept the null hypothesis asserting conformance between the simulator and our
implementation.

\red{
There were occaional instances where we observed p-values smaller than 0.05.  However, careful and
rigorous evaluation is necessary to confirm whether these instances truly indicate a physical
semantics error. (Confirmed in small-scale experiments)
}




\section{Comparisons to Previous Works}

This section emphasizes the potential for utilizing novel approaches for testing quantum simulators
that have not yet been explored.

\subsection{Problems}
Earlier methods for testing quantum simulators, such as QDiff and MorphQ, have employed two-sample
tests, specifically the Kolmogorov-Smirnov test (K-S test).
However, there are two key issues with this approach.

\begin{enumerate}
  \item
    Two-sample testing is performed without the ideal probability distribution that the simulation
    results should align with, which inherently decreases the test's accuracy.
  \item
    The K-S test is fundamentally a statistical test for a continuous probability distribution,
    which is not a suitable fit for our case where we are dealing with a discrete probability
    distribution. QDiff employs a variant of the K-S test for discrete instances, but they don't
    compute the p-value for the test; instead, they use it to determine "\textit{how many
    measurements are needed for a reliable evaluation to ensure the relative error between two
    distributions is within a given threshold t with confidence p?}" On the other hand, MorphQ
    computes the p-value using the standard K-S test for a continuous probability distribution without
    any modifications, but the relevance of the p-value obtained this way remains ambiguous.
\end{enumerate}

\subsection{One-sample testing}

The results produced by our implementation, represented as a discrete probability distribution,
enable the use of one-sample testing, effectively addressing both aforementioned issues.

\begin{enumerate}
  \item
    As we have knowledge of the correct expected probability distribution, we can conduct one-sample
    testing that is more capable of identifying semantic errors.
  \item
    The Chi-square goodness-of-fit test is designed for discrete probability distributions and is more
    suited to our case. It provides a p-value, essentially answering the question, ``Does it make sense
    to say that this distribution originated from this particular probability distribution?".
\end{enumerate}

\subsection{Effectiveness Comparison Between One-sample and Two-sample Testing}
To verify that one-sample testing with the Chi-square goodness-of-fit test is more proficient at
identifying semantic errors compared to two-sample testing with the F-S test, we conducted an
experiment examining each method's ability to identify semantic errors when the semantics of the
Hadamard gate are incorrect:

\begin{lstlisting}
gate h a { U(pi/2 - 0.1, 0,pi) a; } // incorrect: - 0.1
\end{lstlisting}

\red{Graph: As the number of shots increases, the count of p-values less than 0.05 also rises. The
  count of p-values of the one-sample test increases much quicker. Confirmed in small-scale
experiments; this experiment is ongoing.} \\
\\

The graph above demonstrates how the error detection rate, based on p-value, increases with the
number of shots when testing a benchmark suite with incorrect semantics for the Hadamard gate. For
the two-sample test, we used MorphQ's methodology to compute the p-value. We can observe that the
one-sample test identifies a significantly higher number of semantic errors.

Thus, testing a quantum circuit smulator, using our implementation will be more effective than the existing method.

\end{document}

