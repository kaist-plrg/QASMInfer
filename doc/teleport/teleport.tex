\documentclass[10pt,a4paper]{article}
\usepackage[margin=1in]{geometry}
\usepackage{listings}
\usepackage{amsmath}
\usepackage{amsthm}
\usepackage{amsfonts}
\usepackage{enumerate}
\usepackage{amssymb}
\usepackage{graphicx}
\usepackage{booktabs}
\usepackage{xcolor}
\usepackage{stackengine}
\usepackage{quantikz}
\usepackage{cancel}


\lstset{
  basicstyle= \footnotesize \ttfamily,
  % numbers=left,
  % numberstyle=\small,
  % numbersep=8pt,
  showstringspaces=false,
  breaklines=true,
  frame=lines,
  columns=fullflexible,
  backgroundcolor=\color{white},
  emphstyle=\color{blue},
  keywordstyle=\color{brown},
  stringstyle=\color{red},
  commentstyle=\color{gray},
}
\setlength\parindent{0pt}

\newcommand{\<}{\langle}
\renewcommand{\>}{\rangle}
\newcommand{\norm}[1]{\left\lVert #1 \right\rVert}
\newcommand{\abs}[1]{\left\vert #1 \right\vert}
\newcommand{\prths}[1]{\left( #1 \right)}
\newcommand{\braces}[1]{\left\{ #1 \right\}}
\newcommand{\brackets}[1]{\left[ #1 \right]}
\newcommand{\chevrons}[1]{\left\< #1 \right\>}
\newcommand{\prob}[2][]{\underset{#1}{\mathbb{P}}\left[ #2 \right]}
\newcommand{\set}[2]{\left\{ #1 \; \middle\vert \; #2 \right\}}
\newcommand{\tr}[1]{\textrm{tr} \left( #1 \right)}
% \newcommand{\bra}[1]{\left\< #1 \right\vert}
% \newcommand{\ket}[1]{\left\vert #1 \right\>}

\title{Coq Implementation}
\author{Kanguk Lee}
\date{\today}

\begin{document}
% \maketitle

\section{Quantum Teleportation Source Code}

\begin{lstlisting}
OPENQASM 2.0;

gate x a { U(pi,0,pi) a; }
gate z a { U(0,0,pi) a; }
gate h a { U(pi/2,0,pi) a; }

qreg q[3];
creg c0[1];
creg c1[1];
creg c2[1];

// we are trying to send q[0]

// step 0
// prepare an arbitrary qubit
U(0.3,0.2,0.1) q[0];

// step 1
// make a Bell pair
h q[1];
CX q[1],q[2];

// step 2
CX q[0],q[1];
h q[0];

// step 3
measure q[0] -> c0[0];
measure q[1] -> c1[0];

// step 4
if(c0==1) z q[2];
if(c1==1) x q[2];

// now q[0] teleported to q[2]
\end{lstlisting}

\section{Quantum Teleportation Circuit}

\begin{quantikz}[wire types={q,q,q,c,c}]
  \lstick{$q_0:\ket{\psi}$} & \qw & \slice{2} & \ctrl{1} & \gate{H} & \meter{} \wire[d][3]{c} &&&&\\
  \lstick{$q_1:\ket{0}$} & \gate{H} & \ctrl{1} & \targ{} & \slice{3} && \meter{} \wire[d][3]{c}&&&\\
  \lstick{$q_2:\ket{0}$} && \targ{} &&&&& \gate{Z} & \gate{X} & \rstick{$\ket{\psi}$}\\
  \lstick{$c_1$} &&&&&& \slice{4} & \ctrl[vertical wire=c]{-1} &&\\
  \lstick{$c_2$} \slice{1} &&&&&&&& \ctrl[vertical wire=c]{-2} &
\end{quantikz}

\section{Formal Presentation}
\subsection{Preliminaries}
\begin{align*}
  H \ket{0} & = \frac{1}{\sqrt{2}} \prths{\ket{0} + \ket{1}} \\
  H \ket{1} & = \frac{1}{\sqrt{2}} \prths{\ket{0} - \ket{1}} \\
  X \ket{0} & = \ket{1} \\
  X \ket{1} & = \ket{0} \\
  Z \ket{0} & = \ket{0} \\
  Z \ket{1} & = -\ket{1} \\
  C_X \ket{00} & = \ket{00} \\
  C_X \ket{01} & = \ket{01} \\
  C_X \ket{10} & = \ket{11} \\
  C_X \ket{11} & = \ket{10} \\
\end{align*}
\subsection{Step 0}
Alice is the one who owns $q_0 = \ket{\psi}$ and $q_1$ and she wants to teleport $\ket{psi}$ to Bob,
who owns $q_2$.
The teleportation protocol begins with a quantum state or qubit $\ket{\psi}$, in Alice's
possession, that she wants to convey to Bob. This qubit can be written generally as:

$$
\ket{\psi}_{q_0} = \alpha \ket{0}_{q_0} + \beta \ket{1}_{q_0}
$$

The subscript $0$ above is used only to distinguish this qubit states in the first, second and third
horizontal rows of the circuit.

\subsection{Step 1}
Next, the protocol requires that Alice and Bob share a maximally entangled state. This state is
fixed in advance, by mutual agreement between Alice and Bob, and can be any one of the four Bell
states. In this circuit, Alice and Bob will share following Bell state:

$$
\ket{\Phi^+}_{q_1q_2} =
\frac{1}{\sqrt{2}}
\prths{
\ket{0}_{q_1} \otimes \ket{0}_{q_2} +
\ket{1}_{q_1} \otimes \ket{1}_{q_2}
}
$$

by the process we are about to discuss. Given $q_1$ and $q_2$ both initialized to $\ket{0}$:

$$
\textrm{Inital $q_1$ and $q_2$} = \ket{0}_{q_1} \otimes \ket{0}_{q_2} \, ,
$$

the Hadamard gate is applied to $q_1$:

\begin{align*}
\prths{
H \otimes I
} \cdot \prths {
\ket{0}_{q_1} \otimes \ket{0}_{q_2}
} & =
\prths {H \ket{0}_{q_1}} \otimes
\prths {I \ket{0}_{q_2}} \\ & =
\frac{1}{\sqrt{2}}
\prths {\ket{0}_{q_1} + \ket{1}_{q_1}} \otimes
\ket{0}_{q_2} \\ & =
\frac{1}{\sqrt{2}}
\prths {\ket{00}_{q_1q_2} + \ket{10}_{q_1q_2}}
\end{align*}

then $C_X$ (CNOT) gate, $q1$ as a control qubit and $q2$ as a target qubit:

$$
\frac{1}{\sqrt{2}}
\prths {\ket{00}_{q_1q_2} + \ket{10}_{q_1q_2}} \cdot C_X =
\frac{1}{\sqrt{2}}
\prths {\ket{00}_{q_1q_2} + \ket{11}_{q_1q_2}}
$$

At this point, Alice has two particles ($q_0$, the one she wants to teleport, and $q_1$, one of the
entangled pair), and Bob has one particle, $q_2$. In the total system, the state of these three
particles is given by

\begin{align*}
\ket{\psi}_{q_0} \otimes
\frac{1}{\sqrt{2}}
\prths {\ket{00}_{q_1q_2} + \ket{11}_{q_1q_2}} & =
\prths {\alpha \ket{0}_{q_0} + \beta \ket{1}_{q_0}} \otimes
\frac{1}{\sqrt{2}}
\prths {\ket{00}_{q_1q_2} + \ket{11}_{q_1q_2}} \\ & =
\frac{1}{\sqrt{2}}
\prths {
\alpha \ket{000}+
\alpha \ket{011}+
\beta \ket{100}+
\beta \ket{111}
}.
\end{align*}

\subsection{Step 2}

In step 2, we first apply $C_X$ (CNOT) gate with $q_0$ as a control qubit and $q_1$ as a target qubit:

\begin{align*}
\prths{C_X \otimes I } \cdot
\frac{1}{\sqrt{2}}
\prths {
\alpha \ket{000}+
\alpha \ket{011}+
\beta \ket{100}+
\beta \ket{111}
} \\ =
\frac{1}{\sqrt{2}}
\prths {
\alpha \ket{000}+
\alpha \ket{011}+
\beta \ket{110}+
\beta \ket{101}
}
\end{align*}

then $H$ gate to $q_0$:

\begin{align*}
& \prths{H \otimes I } \cdot
\frac{1}{\sqrt{2}}
\prths {
\alpha \ket{000}+
\alpha \ket{011}+
\beta \ket{100}+
\beta \ket{111}
} \\ = &
\frac{1}{2}
\prths {
 \alpha \ket{000}
+\alpha \ket{100}
+\alpha \ket{011}
+\alpha \ket{111}
+\beta  \ket{010}
-\beta  \ket{110}
+\beta  \ket{001}
-\beta  \ket{101}
} \\ = &
\frac{1}{2}
\prths {
 \alpha \ket{000}
+\beta  \ket{001}
+\beta  \ket{010}
+\alpha \ket{011}
+\alpha \ket{100}
-\beta  \ket{101}
-\beta  \ket{110}
+\alpha \ket{111}
}.
\end{align*}

\subsection{Step 3}
Alice will then make two local measurement of $q_0$ and $q_1$. The measurement causes each qubit to
collapse to non-superposed state. There are a total of four possible results of step 3, depending
on the measurement outcomes. (Omit details about normalization.)

\subsubsection{ $q_0 \rightarrow c_0 = 0$, $q_1 \rightarrow c_1 = 0$}
\begin{align*}
\textrm{resulting state}
& =
\frac{1}{2}
\prths {
 \alpha \ket{000}
+\beta  \ket{001}
+\cancel{\beta  \ket{010}}
+\cancel{\alpha \ket{011}}
+\cancel{\alpha \ket{100}}
-\cancel{\beta  \ket{101}}
-\cancel{\beta  \ket{110}}
+\cancel{\alpha \ket{111}}
}
\\ & =
 \alpha \ket{000}
+\beta  \ket{001}
\end{align*}

\subsubsection{ $q_0 \rightarrow c_0 = 0$, $q_1 \rightarrow c_1 = 1$}
\begin{align*}
\textrm{resulting state}
& =
\frac{1}{2}
\prths {
\cancel{\alpha \ket{000}}
+\cancel{\beta  \ket{001}}
+\beta  \ket{010}
+\alpha \ket{011}
+\cancel{\alpha \ket{100}}
-\cancel{\beta  \ket{101}}
-\cancel{\beta  \ket{110}}
+\cancel{\alpha \ket{111}}
}
\\ & =
 \beta  \ket{010}
+\alpha \ket{011}
\end{align*}

\subsubsection{ $q_0 \rightarrow c_0 = 1$, $q_1 \rightarrow c_1 = 0$}
\begin{align*}
\textrm{resulting state}
& =
\frac{1}{2}
\prths {
 \cancel{\alpha \ket{000}}
+\cancel{\beta  \ket{001}}
+\cancel{\beta  \ket{010}}
+\cancel{\alpha \ket{011}}
+\alpha \ket{100}
-\beta  \ket{101}
-\cancel{\beta  \ket{110}}
+\cancel{\alpha \ket{111}}
}
\\ = &
\alpha \ket{100}
-\beta  \ket{101}
\end{align*}

\subsubsection{ $q_0 \rightarrow c_0 = 1$, $q_1 \rightarrow c_1 = 1$}
\begin{align*}
\textrm{resulting state}
& =
\frac{1}{2}
\prths {
 \cancel{\alpha \ket{000}}
+\cancel{\beta  \ket{001}}
+\cancel{\beta  \ket{010}}
+\cancel{\alpha \ket{011}}
+\cancel{\alpha \ket{100}}
-\cancel{\beta  \ket{101}}
-\beta  \ket{110}
+\alpha \ket{111}
}
\\ = &
-\beta  \ket{110}
+\alpha \ket{111}
\end{align*}

\subsection{Step 4}
The result of Alice's measurement tells her which of the above four states the system is in. She can
now send her result to Bob through a classical channel, $c_0$ and $c_1$. These two classical bits
can communicate which of the four results she obtained.

After Bob receives the message from Alice, he will know which of the four states his particle is in.
using this information, he performs a unitary operation on his particle to transform it to the
desired state $\alpha \ket{0}_{q_2} + \beta \ket{1}_{q_2}$.

From now on, we only care about Bob's qubit $q_2$ hence omit the states for $q_0$ and $q_1$.

\subsubsection{ $q_0 \rightarrow c_0 = 0$, $q_1 \rightarrow c_1 = 0$}
\begin{align*}
\textrm{state after step 3}
& =
\alpha \ket{0}_{q_2}
+\beta  \ket{1}_{q_2}
\\ & = \ket{\psi}_{q_2}
\end{align*}

This is exactly the state we want. As with the circuit, Bob does not have to do anything.

\subsubsection{ $q_0 \rightarrow c_0 = 0$, $q_1 \rightarrow c_1 = 1$}
\begin{align*}
\textrm{state after step 3}
& =
\beta  \ket{0}_{q_2}
+\alpha \ket{1}_{q_2}
\end{align*}

Since $c_1$ is 1, Bob just needs to do the $X$ operation as represented in the circuit to get the
state we want:

\begin{align*}
X \cdot \prths {\textrm{state after step 3}}
& =
X \cdot \prths{
\beta  \ket{0}_{q_2}
+\alpha \ket{1}_{q_2}
}
\\ & =
\alpha \ket{0}_{q_2}
+\beta  \ket{1}_{q_2}
\\ & = \ket{\psi}_{q_2}
\end{align*}

\subsubsection{ $q_0 \rightarrow c_0 = 1$, $q_1 \rightarrow c_1 = 0$}
\begin{align*}
\textrm{state after step 3}
& =
\alpha \ket{0}_{q_2}
-\beta  \ket{1}_{q_2}
\end{align*}

Since $c_0$ is 1, Bob just needs to do the $Z$ operation as represented in the circuit to get the
state we want:

\begin{align*}
Z \cdot \prths {\textrm{state after step 3}}
& =
Z \cdot \prths{
\alpha \ket{0}_{q_2}
-\beta  \ket{1}_{q_2}
}
\\ & =
\alpha \ket{0}_{q_2}
+\beta  \ket{1}_{q_2}
\\ & = \ket{\psi}_{q_2}
\end{align*}

\subsubsection{ $q_0 \rightarrow c_0 = 1$, $q_1 \rightarrow c_1 = 1$}
\begin{align*}
\textrm{state after step 3}
& =
-\beta  \ket{0}_{q_2}
+\alpha \ket{1}_{q_2}
\end{align*}
Since both $c_0$ and $c_1$ are 1, Bob needs to do the $Z$ then also $X$ operation as represented in
the circuit to get the state we want:

\begin{align*}
X \cdot Z \cdot \prths {\textrm{state after step 3}}
& =
X \cdot Z \cdot \prths{
-\beta  \ket{0}_{q_2}
+\alpha \ket{1}_{q_2}
}
\\ & =
X \cdot \prths{
-\beta  \ket{0}_{q_2}
-\alpha \ket{1}_{q_2}
}
\\ & =
\prths{
-\alpha \ket{0}_{q_2}
-\beta  \ket{1}_{q_2}
}
\\ & =
-\prths{
\alpha \ket{0}_{q_2}
+\beta  \ket{1}_{q_2}
}
\\ & = -\ket{\psi}_{q_2}
\end{align*}

The global phase in quantum mechanics does not matter because it does not affect observable
properties of the system due to the cancellation effect in the computation of expectation values.
Therefore, we can say that $\ket{\psi}$ and $-\ket{\psi}$ are same.

\end{document}

